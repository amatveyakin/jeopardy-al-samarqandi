\documentclass[a4paper,10pt]{article}
\usepackage{cmap}
\usepackage[utf8]{inputenc}
\usepackage[english,russian]{babel}
\usepackage{a4wide}
\usepackage{indentfirst}
\usepackage{amsmath}
\usepackage{amssymb}
\usepackage[colorlinks]{hyperref}

% hyperref options
\hypersetup{linkcolor = blue}    % Цвет текста ссылок на мишени внутри документа; по умолчанию --- red.
\hypersetup{filecolor = cyan}    % Цвет текста ссылок на локальные PDF файлы; по умолчанию --- cyan.
\hypersetup{citecolor = green}   % Цвет библиографических ссылок, которые печатает команда \cite; по умолчанию --- green.
\hypersetup{urlcolor  = magenta} % Цвет текста URL-ссылок; по умолчанию --- magenta.
\hypersetup{unicode   = true}

\renewcommand{\le}{\leqslant}
\renewcommand{\ge}{\geqslant}

\makeatletter

\renewcommand{\@listI}{%
\leftmargin=40pt
\rightmargin=0pt
\labelsep=5pt
\labelwidth=20pt
\itemindent=0pt
\listparindent=0pt
\topsep=4pt plus 1pt minus 1pt
\partopsep=0pt
\parsep=0pt
\itemsep=2pt plus 1pt minus 1pt}

\renewcommand{\@listii}{%
\leftmargin=25pt
\rightmargin=0pt
\labelsep=5pt
\labelwidth=20pt
\itemindent=0pt
\listparindent=0pt
\topsep=1pt plus 1pt minus 1pt
\partopsep=0pt
\parsep=0pt
\itemsep=1pt plus 1pt minus 1pt}

\makeatother

% % ``а) б) в) ...'' style instead of ``(a) (b) (c) ...'' for second-level lists
% \renewcommand{\theenumii}{\asbuk{enumii}}
% \renewcommand{\labelenumii}{\theenumii)}
%
% \newcounter{savedenumi}
% \newcommand{\saveenumi}{\setcounter{savedenumi}{\value{enumi}}}
% \newcommand{\loadenumi}{\setcounter{enumi}{\value{savedenumi}}}

\newenvironment{topic}{\begin{enumerate}}{\end{enumerate}}
\newcommand{\question}[3]{\item[#1.] #2 \par (Ответ: #3)}


\begin{document}


\section{-АВО- (unfinished)}

\begin{topic}
 \question{0--10}{}{авокадо}
 \question{10--20}{Что-нибудь про <<В стране невыученных уроков>>}{травоядное}
 \question{10--20}{}{наводнение}
 \question{20--30}{в театре/в фонетическом алфавите}{браво}
 \question{20--30}{}{Давос}
 \question{40}{}{}
 \question{50}{Нужна цитата. Возможные источники: \par
  <<Гавриил Троепольский. Белый Бим Черное ухо>> \par
  <<Иван Сергеевич Тургенев. Муму>> \par
  <<Джек Лондон. Белый Клык>> \par}
  {навострить уши}
\end{topic}


\section{Музыкальная тема <<тропы (и дороги (?))>> (unfinished)}

\begin{topic}
 \question{10}{Именно этот вопрос Трубадур задаёт сабжу.}{<<Куда ты, тропинка меня привела?>>}
 \question{20--40}{Вопрос про трассу E-95. можно про то, что соединяет сегодняшная трасса E-95 или как теперь называется бывшая E-95. Правда, кажется, это всё "--- баян.}{}
 \question{30--40}{}{<какой-нибудь троп (литературный)>}
 \question{40}{}{}
 \question{50}{}{}
\end{topic}


\section{Ошибка вышла}

% Во время Второй Мировой войны британские бомбардировщики ежедневно отправлялись бомбить немцев. Большинство самолетов не возвращалось домой, а те, что возвращались, были покрыты следами от пуль немецких зенитных орудий и истребителей.
% Желая увеличить вероятность возвращения экипажей домой, британские инженеры изучали расположение следов от пуль. Инженеры считали, что в тех местах, где больше всего попаданий, нужно дополнительно бронировать корпус самолета. Разумеется, была найдена закономерность: много следов располагалось на крыльях, хвосте и в районе хвостового пулемета. На кабине пилота и топливных баках следов было мало.
% Логично заключить, что нужно добавить брони в тех местах, где больше всего следов. Но это неверно.
% Самолеты с пулями в кабине пилота и топливных баках не смогли вернуться домой, а на вернувшихся самолетах следы от пуль были найдены как раз в достаточно укрепленных местах. Важная информация была у сбитых самолетов, а не у вернувшихся.

% и другие исторические ляпы

\begin{topic}
 \question{10}{}{}
 \question{20}{}{}
 \question{30}{}{}
 \question{40}{}{}
 \question{50}{}{}
\end{topic}


\end{document}






\section{#}

\begin{topic}
 \question{10}{}{}
 \question{20}{}{}
 \question{30}{}{}
 \question{40}{}{}
 \question{50}{}{}
\end{topic}
