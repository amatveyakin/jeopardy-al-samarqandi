\documentclass[a4paper,10pt]{article}
\usepackage{cmap}
\usepackage[utf8]{inputenc}
\usepackage[english,russian]{babel}
\usepackage{a4wide}
\usepackage{indentfirst}
\usepackage{amsmath}
\usepackage{amssymb}
\usepackage{changepage}
\usepackage[colorlinks]{hyperref}

% hyperref options
\hypersetup{linkcolor = blue}    % Цвет текста ссылок на мишени внутри документа; по умолчанию --- red.
\hypersetup{filecolor = cyan}    % Цвет текста ссылок на локальные PDF файлы; по умолчанию --- cyan.
\hypersetup{citecolor = green}   % Цвет библиографических ссылок, которые печатает команда \cite; по умолчанию --- green.
\hypersetup{urlcolor  = magenta} % Цвет текста URL-ссылок; по умолчанию --- magenta.
\hypersetup{unicode   = true}

\let\keyword\textsc
\newenvironment{topic}{\begin{enumerate}}{\end{enumerate}}
\newcommand{\question}[3]{\item[#1.] #2 \par \keyword{Ответ:} #3}
\newcommand{\alternative}[1]{\par \keyword{В зачёт:} #1}
\newcommand{\commentary}[1]{\par \keyword{Комментарий:} #1}
\newcommand{\topiccommentary}[1]{\begin{adjustwidth}{0.9cm}{} \vspace{-0.3em}\textbf{\hbox to 0pt{\hss{(}}#1)} \end{adjustwidth}}


\begin{document}

\begin{center}
 \Huge Своя игра
 \medskip

 \Large Темы Матвеякина Андрея
\end{center}
\bigskip


\tableofcontents

\newpage
\part{Турнир VI (осень-2013)}

\section{-АВО-}

\begin{topic}
 \question{10}{\emph{Именно эту букву} международного фонетического алфавита можно услышать в театре после хорошего выступления.}{Браво.}
 \question{20}{Герой мультфильма «В стране невыученных уроков» попал в опасное положение из-за того, что ошибочно не причислил коров \emph{именно к этой категории.}}{Травоядные.}
 \question{30}{\emph{Именно так называется} кратковременное и непериодическое увеличение уровня монооксида дигидрогена.}{Наводнение; Паводок.}
 \question{40}{\emph{Именно так назывались} корабли экспедиции Николая Рязанского, давшие в последствии имя советскому произведению музыкального искусства.}{Юнона и Авось.}
 \question{50}{\emph{Именно это сделал} герой Джека Лондона, подбегая к волчице. И не зря: её намерения оказались вовсе не столь дружелюбными, как он надеялся.}{Навострил уши.}\commentary{Речь о романе «Белый клык».}
\end{topic}


\section{Зелёная}

\begin{topic}
 \question{10}{\emph{Именно он} сопровождал профессора Селезнёва и его дочь Алису в экспедиции, направлявшейся на поиски новых видов животных для Московского зоопарка.}{Капитан Зелёный.}
 \question{20}{\emph{Именно такую зелёную фигуру} можно увидеть на светофоре после красного X-образного сигнала.}{Стрелку вниз.}\commentary{Речь идёт о реверсивных светофорах.}
 \question{30}{Подпадающий под название темы синтетический анилиновый краситель, больше известный как антисептическое средство, \emph{официально называется именно так.}}{Бриллиантовый зелёный.}\alternative{Основной зелёный~1; № 42040; Основной ярко-зелёный; Малахитовый зелёный~Ж.; Тетраэтил-4,4-диаминотрифенилметана оксалат.}
 \question{40}{Согласно шутке \emph{этот персонаж} так и не смог выучить порядок цветов в радуге "--- не помогла проверенная годами мнемоника.}{Магистр Йода.}
 \question{50}{\emph{Именно так звали} зелёного короля в сказочной повести Александра Волкова «Семь подземных королей».}{Ментахо.}
\end{topic}


\section{Цепная тема}
\topiccommentary{Ответ на каждый следующий вопрос отличается от предыдущего добавлением, удалением или изменением одной буквы.}

\begin{topic}
 \question{10}{\emph{Именно его} предлагают получить из представителя отряда двукрылых в игре с правилами аналогичными правилам этой темы, а также в известной поговорке.}{Слон.}\commentary{Делать из мухи слона.}
 \question{20}{\emph{Именно их нападение} вынесено в заголовок не то второй, не то пятой части известной кинематографической гексалогии.}{Клон.}\commentary{Фильм «Атака клонов» является 2-м в сюжетном порядке и 5-м в порядке съёмки.}
 \question{30}{В одном из переводов художественного фильма «Назад в будущее» плохо разбирающийся в молодёжном сленге профессор Браун многократно интересовался, \emph{какое он имеет отношение к делу.} Между тем, в оригинальной английской версии Марти говорил вовсе не про него, а про температуру.}{Склон.}\commentary{Английское «Cool» в фильме было переведено как «Круто», на что Браун логично спрашивал: «А при чём здесь склон?»}
 \question{40}{Многие литературные произведения были впервые прочитаны \emph{в одном из них} задолго до появления в печати.}{Салон.}
 \question{50}{Согласно старому анекдоту одна иностранка очень обиделась, когда русский слуга подавал ей \emph{именно этот предмет.}}{Салоп.}\commentary{Произнесённую слугой фразу «ваш салоп» гостья-француженка услышала как «vache salope» "--- «распутная корова», из-за чего и возник скандал.}
\end{topic}


\newpage
\part{Неотыгранные}

\section{Музыкальная тема «тропы (и дороги (?))» (unfinished)}

\begin{topic}
 \question{10}{\emph{Именно этот вопрос} Трубадур задаёт сабжу.}{«Куда ты, тропинка меня привела?».}
 \question{20--40}{Вопрос про трассу E-95. можно про то, что соединяет сегодняшная трасса E-95 или как теперь называется бывшая E-95. Правда, кажется, это всё "--- баян.}{}
 \question{30--40}{}{<какой-нибудь троп (литературный)>.}
 \question{40}{}{}
 \question{50}{}{}
\end{topic}


\section{Ошибка вышла (unfinished)}

\begin{topic}
 \question{??}{<Во время Второй Мировой войны британские бомбардировщики ежедневно отправлялись бомбить немцев. Большинство самолетов не возвращалось домой, а те, что возвращались, были покрыты следами от пуль немецких зенитных орудий и истребителей.

 Желая увеличить вероятность возвращения экипажей домой, британские инженеры изучали расположение следов от пуль. Инженеры считали, что в тех местах, где больше всего попаданий, нужно дополнительно бронировать корпус самолета. Разумеется, была найдена закономерность: много следов располагалось на крыльях, хвосте и в районе хвостового пулемета. На кабине пилота и топливных баках следов было мало.

 Логично заключить, что нужно добавить брони в тех местах, где больше всего следов. Но это неверно.

 Самолеты с пулями в кабине пилота и топливных баках не смогли вернуться домой, а на вернувшихся самолетах следы от пуль были найдены как раз в достаточно укрепленных местах. Важная информация была у сбитых самолетов, а не у вернувшихся.>}{}
 \question{10--20}{<вопрос про одноимённую песню Высоцкого: именно это на самом деле>}{Историю болезни.}
 \question{20}{}{}
 \question{30}{\emph{Именно этот персонаж} появился в результате неправильного перевода сообщения DOS об ошибке чтения диска.}{Генерал Фэйлор.}\commentary{Оригинал сообщения: «General failure reading drive a:».}
 \question{40}{}{}
 \question{50}{}{}
\end{topic}


\section{Настольные игры (unfinished)}

\begin{topic}
 \question{??}{<Про польскую игру «очередь». Можно использовать факт: Вперёд помогают пройти, например, карточки «Мать с маленьким ребёнком» и «Вас тут не стояло»>}{Очередь (Kolejka).}
 \question{0--10}{\emph{Именно столько раз} менялся дизайн рубашки карт «Magic: The Gathering»}{Ноль.}
 \question{10--20}{\emph{Именно такое максимальное количество клеток} может одновременно держать под боем шахматный ферзь.}{27.}
 \question{20--30}{Американский изобретатель и футуролог Рэймонд Курцвейл назвал точку, в которой экспоненциальный рост некоторого фактора начинает оказывать существенное влияние на общую экономическую ситуацию, \emph{второй половиной именно этого,} отсылая нас к древней легенде, в которой плохо разбиравшийся в математике правитель так и не смог исполнить опреметчивое обещание.}{Шахматной доски.}
 \question{40}{}{}
 \question{50}{}{}
\end{topic}


\section{Идеи тем}

\begin{itemize}
 \item http://habrahabr.ru/post/178155/ и http://habrahabr.ru/post/194362/
 \item Тема про метро
\end{itemize}


\section{Отдельные вопросы}

\begin{topic}
 \question{??}{Что-нибудь про графен (прочный; одноатомарный слой; динамики; куб, стоящий на одуванчике...) (можно прямо такую тему и сделать: «удивительные вещества» или «тонкии материи»).}
 \question{40--50}{\emph{Именно по этой причине} пассажиры поездов дальнего следования, проезжая через Горячий Ключ, вынуждены ждать замены локомотива, а путешествующие на электричках "--- и вовсе пересаживаться в другой состав, ждущий их на сосденем пути.}{В Горячем Ключе станции стыкуются два рода тока: переменный со стороны Краснодара и постоянный со стороны побережья.}
\end{topic}

\end{document}






\section{# (unfinished)}

\begin{topic}
 \question{10}{}{}
 \question{20}{}{}
 \question{30}{}{}
 \question{40}{}{}
 \question{50}{}{}
\end{topic}
