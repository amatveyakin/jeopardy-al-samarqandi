\documentclass[a4paper,10pt]{article}
\usepackage{cmap}
\usepackage[utf8]{inputenc}
\usepackage[english,russian]{babel}
\usepackage{a4wide}
\usepackage{indentfirst}
\usepackage{amsmath}
\usepackage{amssymb}
\usepackage{changepage}
\usepackage{chemfig}
\usepackage[colorlinks]{hyperref}

% hyperref options
\hypersetup{linkcolor = blue}    % Цвет текста ссылок на мишени внутри документа; по умолчанию --- red.
\hypersetup{filecolor = cyan}    % Цвет текста ссылок на локальные PDF файлы; по умолчанию --- cyan.
\hypersetup{citecolor = green}   % Цвет библиографических ссылок, которые печатает команда \cite; по умолчанию --- green.
\hypersetup{urlcolor  = black}   % Цвет текста URL-ссылок; по умолчанию --- magenta.
\hypersetup{unicode   = true}

\makeatletter
\@addtoreset{section}{part}
\makeatother

\let\keyword\textsc
\newenvironment{topic}{\begin{enumerate}}{\end{enumerate}}
\newcommand{\question}[3]{\item[#1.] #2 \par \keyword{Ответ:} #3}
\newcommand{\alternative}[1]{\par \keyword{В зачёт:} #1}
\newcommand{\commentary}[1]{\par \keyword{Комментарий:} #1}
\newcommand{\topiccommentary}[1]{\begin{adjustwidth}{0.9cm}{} \vspace{-0.3em}\textbf{\hbox to 0pt{\hss{(}}#1)} \end{adjustwidth}}
\newcommand{\placeholder}{~\par~\par~}
\newcommand{\additional}{$\Im$}


\begin{document}

\begin{center}
 \Huge Своя игра
 \medskip

 \Large Темы Матвеякина Андрея
\end{center}
\bigskip


% TODO: Сделать содержание более компактным
\tableofcontents

\newpage
\part{Турнир VI (осень-2013)}

\section{-АВО-}

\begin{topic}
 \question{10}{\emph{Именно эту букву} международного фонетического алфавита можно услышать в театре после хорошего выступления.}{Браво.}
 \question{20}{Герой мультфильма «В стране невыученных уроков» попал в опасное положение из-за того, что ошибочно не причислил коров \emph{именно к этой категории.}}{Травоядные.}
 \question{30}{\emph{Именно так называется} кратковременное и непериодическое увеличение уровня монооксида дигидрогена.}{Наводнение; Паводок.}
 \question{40}{\emph{Именно так назывались} корабли экспедиции Николая Рязанского, давшие в последствии имя советскому произведению музыкального искусства.}{Юнона и Авось.}
 \question{50}{\emph{Именно это сделал} герой Джека Лондона, подбегая к волчице. И не зря: её намерения оказались вовсе не столь дружелюбными, как он надеялся.}{Навострил уши.}\commentary{Речь о романе «Белый клык».}
\end{topic}


\section{Зелёная}

\begin{topic}
 \question{10}{\emph{Именно он} сопровождал профессора Селезнёва и его дочь Алису в экспедиции, направлявшейся на поиски новых видов животных для Московского зоопарка.}{Капитан Зелёный.}
 \question{20}{\emph{Именно такую зелёную фигуру} можно увидеть на светофоре после красного X-образного сигнала.}{Стрелку вниз.}\commentary{Речь идёт о реверсивных светофорах.}
 \question{30}{Подпадающий под название темы синтетический анилиновый краситель, больше известный как антисептическое средство, \emph{официально называется именно так.}}{Бриллиантовый зелёный.}\alternative{Основной зелёный~1; № 42040; Основной ярко-зелёный; Малахитовый зелёный~Ж.; Тетраэтил-4,4-диаминотрифенилметана оксалат.}
 \question{40}{Согласно шутке \emph{этот персонаж} так и не смог выучить порядок цветов в радуге "--- не помогла проверенная годами мнемоника.}{Магистр Йода.}
 \question{50}{\emph{Именно так звали} зелёного короля в сказочной повести Александра Волкова «Семь подземных королей».}{Ментахо.}
\end{topic}


\section{Цепная тема}
\topiccommentary{Ответ на каждый следующий вопрос отличается от предыдущего добавлением, удалением или изменением одной буквы.}

\begin{topic}
 \question{10}{\emph{Именно его} предлагают получить из представителя отряда двукрылых в игре с правилами аналогичными правилам этой темы, а также в известной поговорке.}{Слон.}\commentary{Делать из мухи слона.}
 \question{20}{\emph{Именно их нападение} вынесено в заголовок не то второй, не то пятой части известной кинематографической гексалогии.}{Клон.}\commentary{Фильм «Атака клонов» является 2-м в сюжетном порядке и 5-м в порядке съёмки.}
 \question{30}{В одном из переводов художественного фильма «Назад в будущее» плохо разбирающийся в молодёжном сленге профессор Браун многократно интересовался, \emph{какое он имеет отношение к делу.} Между тем, в оригинальной английской версии Марти говорил вовсе не про него, а про температуру.}{Склон.}\commentary{Английское «Cool» в фильме было переведено как «Круто», на что Браун логично спрашивал: «А при чём здесь склон?»}
 \question{40}{Многие литературные произведения были впервые прочитаны \emph{в одном из них} задолго до появления в печати.}{Салон.}
 \question{50}{Согласно старому анекдоту одна иностранка очень обиделась, когда русский слуга подавал ей \emph{именно этот предмет.}}{Салоп.}\commentary{Произнесённую слугой фразу «ваш салоп» гостья-француженка услышала как «vache salope» "--- «распутная корова», из-за чего и возник скандал.}
\end{topic}


\newpage
\part{Турнир VII (весна-2014)}

\section{Ошибка вышла (unfinished)}

\begin{topic}
 \question{10}{<Волан-де-морт: правильно ли выбрал мальчика?>}{}
 \question{20}{<Во время Второй Мировой войны британские бомбардировщики ежедневно отправлялись бомбить немцев. Большинство самолетов не возвращалось домой, а те, что возвращались, были покрыты следами от пуль немецких зенитных орудий и истребителей.

 Желая увеличить вероятность возвращения экипажей домой, британские инженеры изучали расположение следов от пуль. Инженеры считали, что в тех местах, где больше всего попаданий, нужно дополнительно бронировать корпус самолета. Разумеется, была найдена закономерность: много следов располагалось на крыльях, хвосте и в районе хвостового пулемета. На кабине пилота и топливных баках следов было мало.

 Логично заключить, что нужно добавить брони в тех местах, где больше всего следов. Но это неверно.

 Самолеты с пулями в кабине пилота и топливных баках не смогли вернуться домой, а на вернувшихся самолетах следы от пуль были найдены как раз в достаточно укрепленных местах. Важная информация была у сбитых самолетов, а не у вернувшихся.>

 [оказался баян, можно придумать другой факт про систематическую ошибку выжившего]}{}
 \question{30}{<вопрос про одноимённую песню Высоцкого: именно это на самом деле>}{Историю болезни.}
 \question{40}{\emph{Именно этот персонаж} появился в результате неправильного перевода сообщения DOS об ошибке доступа к диску.}{Генерал Фэйлор.}\commentary{Оригинал сообщения: «General failure reading drive a:».}
 \question{50}{<Про БЧХ (надо не налажать, и дать описание под которое ничего кроме них не подходит)>}{Коды Боуза–Чоудхури–Хокенгема}
\end{topic}


\newpage
\section{Палиндромы}

\begin{topic}
 \question{\additional}{\emph{Именно такой} будет ближайшая дата-палиндром формата ДД.ММ.ГГГГ.}{2 февраля 2020\,г. (02.02.2020).}
 \question{\additional}{\emph{Именно так} в азбуке Морзе выглядит международный сигнал бедствия, известный как SOS.}{3~точки "--- 3~тире "--- 3~точки, без каких-либо межбуквенных интервалов.}
 \question{10}{\emph{Именно столько} клавиш имеется на современном фортепиано.}{88.}
 \question{20}{\emph{Именно таков} код HTTP-ответа «Accepted».}{202.}
 \question{30}{\emph{Именно в этом году} родился Герних III, сын Генриха II и Екатерины Медичи, последний король Франции из династии Валуа.}{1551.}
 \question{40}{\emph{Именна так называется} натуральное число, которое не может стать палиндромом с помощью итеративного процесса «перевернуть и сложить» в десятичной системе счисления.}{Число Лишрел (англ. Lychrel number).}
 \question{50}{\emph{Именно так звучит} известный со времён Римской империи палиндром, складывающийся также в магический квадрат и повествующий о жизни сельхозработника.}{SATOR AREPO TENET OPERA ROTAS.}
\end{topic}


\section{Настольные игры (unfinished)}

\begin{topic}
 \question{10}{Именно в эту игру Ноздрёв после долгих усилий всё-таки смог уговорить сыграть Чичикова.}{Шашки.}
 \question{20}{\emph{Именно такое максимальное количество клеток} может одновременно держать под боем шахматный ферзь.}{27.}
 \question{30}{\emph{Именно так называется} единственная покерная комбинация, которую нельзя собрать в Texas Holdem "--- популярной разновидность покера.}{Покер.}\commentary{Покер состоит из пяти карт одного достоинства, а в Texas Holdem играют одной колодой без джокеров.}
 \question{40}{\emph{Именно с такими картами} в некоторых разновидностях преферанса игрок может объявить «преферанс».}{Нужно иметь на руках 10 гарантированных взяток при игре без козыря при любых картах соперников и любом сносе.}\alternative{Тузы, короли и дамы всех мастей.}
 \question{50}{Американский изобретатель и футуролог Рэймонд Курцвейл назвал точку, в которой экспоненциальный рост некоторого фактора начинает оказывать существенное влияние на общую экономическую ситуацию, \emph{второй половиной именно этого,} отсылая нас к древней легенде, в которой правитель не смог исполнить опреметчивое обещание.}{Шахматной доски.}
\end{topic}


\newpage
\part{Неотыгранные}

\section{-ИНТ- (unfinished)}

\begin{topic}
 \question{??}{<анекдот про интеграл> / <история с Сыроежкиным>}{Интеграл.}
 \question{??}{<перевод «Let's get dangerous»>}{От винта.}
 \question{??}{\placeholder}{Принтер}
 \question{??}{\placeholder}{Финт ушами}
 \question{??}{\placeholder}{Плинтус}
 \question{??}{\placeholder}{Пинта}
 \question{??}{\placeholder}{}
\end{topic}


\section{Чувства (unfinished)}

% Сначала вопросы про обычные чувства, в конце — не очень

\begin{topic}
 \question{??}{Что-нибудь про Фемиду\placeholder}{}
 \question{??}{\placeholder}{Со сладким чувством победы, с горьким чувством вины.}
 \question{??}{\placeholder}{Чувство локтя}
 \question{??}{\placeholder}{}
 \question{??}{\placeholder}{}
\end{topic}


\section{Музыкальная тема «тропы (и дороги (?))» (unfinished)}

\begin{topic}
 \question{10}{\emph{Именно этот вопрос} Трубадур задаёт сабжу.}{«Куда ты, тропинка меня привела?».}
%  \question{20--40}{Вопрос про трассу E-95. можно про то, что соединяет сегодняшная трасса E-95 или как теперь называется бывшая E-95. Правда, кажется, это всё "--- баян.}{}  % TODO: Может, выкинуть её? Ведь все ждут, скучно прямо. Или по крайней мере, придумать нетривиальный вопрос
 \question{30--40}{}{<какой-нибудь троп (литературный)>.}
 \question{??}{... Тропа, где встретишь питекантропа...}{}
 \question{??}{}{Elvenpath}
 \question{??}{\placeholder}{}
 \question{??}{\placeholder}{}
 \question{??}{\placeholder}{}
\end{topic}


\section{Закончите высказывание}
\topiccommentary{Вы должны закончить пословицу, афоризм или цитату. Зачёт по смыслу.}

\begin{topic}
 \question{??}{Если вы самый умный человек в комнате,~...}{...~то вы не в той комнате, где должны находиться.}
 \question{??}{}{}
 \question{??}{}{}
 \question{??}{}{}
 \question{??}{}{}
\end{topic}


\newpage
\section{Идеи тем}

\begin{itemize}
 \item http://habrahabr.ru/post/178155/ и http://habrahabr.ru/post/194362/
 \item Тема про метро
 \item «Оксюмороны»
 \item «Верны традициям»: вопросы вида «Именно с такого-то года не менялось X»
 \item Угадай мелодия по этнической музыке. Нужно угадать нацию.
 \item Шуточная/детская/игры слов. Вопросы в стиле «Куда пролетает пролетариат?»
 \item Смешные аббревиатуры: «Жизнеобеспечение пилотируемых аппаратов», ЛЕНАЭРОПРОЕКТ, ...
       (а) Либо просто аббревиатуры: BACH (си-бемоль, ля, до, си), ...
       (б) Либо misheard lirycs и т.п.: «May the FOURTH», ...
\end{itemize}


\section{Отдельные вопросы}

\begin{topic}
 \question{??}{www -> uuuuuu -> 6 (3 по 2) you (photo) — индивидуальные раздатки}
 \question{??}{Что-нибудь про графен (прочный; одноатомарный слой; динамики; куб, стоящий на одуванчике...) (можно прямо такую тему и сделать: «удивительные вещества» или «тонкии материи»).}
 \question{40--50}{\emph{Именно по этой причине} пассажиры поездов дальнего следования, проезжая через Горячий Ключ, вынуждены ждать замены локомотива, а путешествующие на электричках "--- и вовсе пересаживаться в другой состав, ждущий на сосденем пути.}{В Горячем Ключе станции стыкуются два рода тока: переменный со стороны Краснодара и постоянный со стороны побережья.}
 \question{??}{Голос из «Федота-стрельца...» был готов идти даже \emph{туда,} ради того, чтобы попасть в коллектив}{К пчёлам в улей.}
 \question{??}{\emph{Именно с ним} сравнивали рок-н-ролл Алиса, ДДТ, Кино, Наутилус (TODO: найти что-то нибудь общее в песнях про рок-н-ролл хотя бы у подмножества)}{?}
 \question{??}{Из «Шелока»: город, про нападение на которые командование знало, но решило отдать его врагу, чтобы не раскрывать информатора}{}
 \question{??}{Когда нам душно, это обычно не недостаток кислорода, а \emph{он.}  TODO: переформулировать вопрос; [можно в «чувства»]}{избыток \chemfig{CO_2}; TODO: проверить факт}
 \question{??}{Какой-нибудь вопрос про «Pale blue dot», например: «Именно этот объёкт занимал на ... 0,12 пикселя (Земля)». Можно следать тему «голубая» (ещё вопрос: какого цвета в оригинале щенок) или тему «знаменитые фотографии»}{}{}
 \question{??}{Из TBBT: самое распространённое название улицы?}{Second  TODO: проверить}\commentary{Улицу First обычно потом переименовывают в главную и т.\,п.}
 \question{??}{http://en.wikipedia.org/wiki/Assassination\_market, http://habrahabr.ru/post/202654}{}
 \question{??}{}{Валовое национальное счастье (Gross National Happiness)}
 \question{??}{http://habrahabr.ru/post/222707, http://en.wikipedia.org/wiki/Bombardier\_beetle}{}
 \question{??}{Про «Доктрину крепости»}{}
 \question{??}{Именно это устройство по мнению xkcd отделяет мальчиков от мужчин}{Центрифуга}\commentary{\url{http://store-xkcd-com.myshopify.com/products/centrifuge}}
 \question{??}{Привести пример слова хотя бы с тремя твёрдыми или мягкими знаками [TODO: более чёткая фомулировка, чтобы было понятно, что значи могут быть перемешаны] (Может, тема: «Приведите пример». И следить, чтобы было хотя бы два–три ответа)}{Фельдъегерь}
 \question{??}{Про то, почему у бутылок сужается горло? [TODO: fact-check]}{Пробка стоит дорого}
 \question{??}{После того как разработка российского микропроцессора Эльбрус 2000 затянулась, проект был переименован по классическому шаблону, что привело к распространению шутки о том, что процессор будет выпушен \emph{именно в этом году.}  [Источник: Илья; TODO: fact-check]}{2048.}\commentary{Переименовали в E2K.})
 \question{??}{Вопрос или даже тема про сигналы бедствия (Mayday и пр.)}{} % https://ru.wikipedia.org/wiki/%D0%A1%D0%B8%D0%B3%D0%BD%D0%B0%D0%BB_%D0%B1%D0%B5%D0%B4%D1%81%D1%82%D0%B2%D0%B8%D1%8F
 \question{??}{Вопрос про гапакс. Можно тему: называет несколько слов, нужно назвать общий лингвистический термин. (Как отсекать тривиальные варианты вроде: «существительные»?)}{} % https://ru.wikipedia.org/wiki/%D0%93%D0%B0%D0%BF%D0%B0%D0%BA%D1%81
\end{topic}

\end{document}






\section{# (unfinished)}

 \question{??}{}{}

\begin{topic}
 \question{10}{}{}
 \question{20}{}{}
 \question{30}{}{}
 \question{40}{}{}
 \question{50}{}{}
\end{topic}
