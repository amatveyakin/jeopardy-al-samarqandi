\documentclass[a4paper,10pt]{article}
\usepackage{cmap}
\usepackage[utf8]{inputenc}
\usepackage[english,russian]{babel}
\usepackage{a4wide}
\usepackage{indentfirst}
\usepackage{amsmath}
\usepackage{amssymb}
\usepackage{changepage}
\usepackage[colorlinks]{hyperref}

% hyperref options
\hypersetup{linkcolor = blue}    % Цвет текста ссылок на мишени внутри документа; по умолчанию --- red.
\hypersetup{filecolor = cyan}    % Цвет текста ссылок на локальные PDF файлы; по умолчанию --- cyan.
\hypersetup{citecolor = green}   % Цвет библиографических ссылок, которые печатает команда \cite; по умолчанию --- green.
\hypersetup{urlcolor  = magenta} % Цвет текста URL-ссылок; по умолчанию --- magenta.
\hypersetup{unicode   = true}

\let\keyword\textsc
\newenvironment{topic}{\begin{enumerate}}{\end{enumerate}}
\newcommand{\question}[3]{\item[#1.] #2 \par \keyword{Ответ:} #3}
\newcommand{\alternative}[1]{\par \keyword{В зачёт:} #1}
\newcommand{\commentary}[1]{\par \keyword{Комментарий:} #1}
\newcommand{\topiccommentary}[1]{\begin{adjustwidth}{0.9cm}{} \vspace{-0.3em}\textbf{\hbox to 0pt{\hss{(}}#1)} \end{adjustwidth}}


\begin{document}


\section{-АВО- (unfinished)}

\begin{topic}
 \question{10--20}{Что-нибудь про «В стране невыученных уроков»}{травоядное}
 \question{10--20}{}{наводнение}
 \question{20--30}{в театре/в фонетическом алфавите}{браво}
 \question{20--30}{}{Давос}
 \question{40}{}{}
 \question{50}{Нужна цитата. Возможные источники: \par
  «Гавриил Троепольский. Белый Бим Черное ухо» \par
  «Иван Сергеевич Тургенев. Муму» \par
  «Джек Лондон. Белый Клык» \par}
  {навострить уши}
\end{topic}


\section{Музыкальная тема «тропы (и дороги (?))» (unfinished)}

\begin{topic}
 \question{10}{\emph{Именно этот вопрос} Трубадур задаёт сабжу.}{«Куда ты, тропинка меня привела?»}
 \question{20--40}{Вопрос про трассу E-95. можно про то, что соединяет сегодняшная трасса E-95 или как теперь называется бывшая E-95. Правда, кажется, это всё "--- баян.}{}
 \question{30--40}{}{<какой-нибудь троп (литературный)>}
 \question{40}{}{}
 \question{50}{}{}
\end{topic}


\section{Ошибка вышла (unfinished)}

\begin{topic}
 \question{??}{<Во время Второй Мировой войны британские бомбардировщики ежедневно отправлялись бомбить немцев. Большинство самолетов не возвращалось домой, а те, что возвращались, были покрыты следами от пуль немецких зенитных орудий и истребителей.

 Желая увеличить вероятность возвращения экипажей домой, британские инженеры изучали расположение следов от пуль. Инженеры считали, что в тех местах, где больше всего попаданий, нужно дополнительно бронировать корпус самолета. Разумеется, была найдена закономерность: много следов располагалось на крыльях, хвосте и в районе хвостового пулемета. На кабине пилота и топливных баках следов было мало.

 Логично заключить, что нужно добавить брони в тех местах, где больше всего следов. Но это неверно.

 Самолеты с пулями в кабине пилота и топливных баках не смогли вернуться домой, а на вернувшихся самолетах следы от пуль были найдены как раз в достаточно укрепленных местах. Важная информация была у сбитых самолетов, а не у вернувшихся.>}{}
 \question{10--20}{<вопрос про одноимённую песню Высоцкого: именно это на самом деле>}{Историю болезни}
 \question{20}{}{}
 \question{30}{\emph{Именно этот персонаж} появился в результате неправильного перевода сообщения DOS об ошибке чтения диска.}{Генерал Фэйлор}\commentary{General failure reading drive a:}
 \question{40}{}{}
 \question{50}{}{}
\end{topic}


\section{Настольные игры (unfinished)}

\begin{topic}
 \question{??}{<Про польскую игру «очередь». Можно использовать факт: Вперёд помогают пройти, например, карточки «Мать с маленьким ребёнком» и «Вас тут не стояло»>}{Очередь (Kolejka)}
 \question{0--10}{\emph{Именно столько раз} менялся дизайн рубашки карт MTG}{Ноль}
 \question{10--20}{\emph{Именно такое максимальное количество клеток} может одновременно держать под боем шахматный ферзь.}{27}
 \question{20--30}{Американский изобретатель и футуролог Рэймонд Курцвейл назвал точку, в которой экспоненциальный рост некоторого фактора начинает оказывать существенное влияние на общую экономическую ситуацию, \emph{второй половиной именно этого,} отсылая нас к древней легенде, в которой плохо разбиравшийся в математике правитель так и не смог исполнить опреметчивое обещание.}{Шахматной доски}
 \question{40}{}{}
 \question{50}{}{}
\end{topic}


\section{Зелёная (unfinished)}

\begin{topic}
 \question{??}{\emph{Именно такой по счёту} была построена зелёная ветка московского метро.}{А вот какой? Номер у неё 2-й, но википедия говорит, что построена она третьей.}
 \question{10--20}{Действие \emph{именно этого флага} отменяется зелёный флаг в автогонках <Возможно, лучше уточнить вариант правил (например, FIA)>}{Жёлтого}
 \question{20--30}{Согласно шутке \emph{этот персонаж} так и не смог выучить порядок цветов в радуге "--- не помогла проверенная годами мнемоника.}{Йода}
 \question{30--40}{Подпадающий под название темы синтетический анилиновый краситель, больше известный как антисептическое средство, \emph{официально назвается именно так.}}{Бриллиантовый зелёный}\alternative{основной зелёный~1; № 42040; основной ярко-зелёный; малахитовый зелёный~Ж.; тетраэтил-4,4-диаминотрифенилметана оксалат}
 \question{40}{}{}
 \question{50}{}{}
\end{topic}


\section{Цепная тема (unfinished)}
\topiccommentary{Ответ на кажый следующий вопрос отличается от предыдущего добавлением, удаланием или изменением одной буквы}

\begin{topic}
 \question{10}{\emph{Именно его} предлагают получить из представителя отряда двукрылых в игре с правилами аналогичными правилам этой темы, а также в известной поговорке.}{Слон}\commentary{Делать из мухи слона.}
 \question{20}{\emph{Именно их нападение} вынесено в заголовок не то второй, не то пятой части известной кинематографической гексалогии.}{Клон}\commentary{Фильм «Атака клонов» является 2-м в сюжетном порядке и 5-м в порядке съёмки.}
 \question{30}{В одном из переводов художественного фильма «Назад в будущее» плохо разбирающийся в молодёжном сленге профессор Браун многократно интересовался, \emph{какое он имеет отношение к делу.} Между тем, в оригинальной английской версии Марти говорил вовсе не про него, а про температуру.}{Склон}\commentary{Английское «Cool!» на русский часто переводят как «Круто!» На что Браун логично спрашивал: «А при чём здесь склон?»}
 \question{40}{}{Салон}
 \question{50}{}{Салоп}  % FIXME: слишком мало переходов из салона
\end{topic}


\section{Идеи тем}

\begin{itemize}
 \item http://habrahabr.ru/post/178155/
 \item Тема про метро
\end{itemize}


\section{Отдельные вопросы}

\begin{topic}
 \question{??}{Что-нибудь про графен (прочный; одноатомарный слой; динамики; куб, стоящий на одуванчике...) (можно прямо такую тему и сделать: «удивительные вещества»)}
\end{topic}

\end{document}






\section{# (unfinished)}

\begin{topic}
 \question{10}{}{}
 \question{20}{}{}
 \question{30}{}{}
 \question{40}{}{}
 \question{50}{}{}
\end{topic}
